% Options for packages loaded elsewhere
% Options for packages loaded elsewhere
\PassOptionsToPackage{unicode}{hyperref}
\PassOptionsToPackage{hyphens}{url}
\PassOptionsToPackage{dvipsnames,svgnames,x11names}{xcolor}
%
\documentclass[
  letterpaper,
  sfsidenotes]{tufte-book}
\usepackage{xcolor}
\usepackage{amsmath,amssymb}
\setcounter{secnumdepth}{-\maxdimen} % remove section numbering
\usepackage{iftex}
\ifPDFTeX
  \usepackage[T1]{fontenc}
  \usepackage[utf8]{inputenc}
  \usepackage{textcomp} % provide euro and other symbols
\else % if luatex or xetex
  \usepackage{unicode-math} % this also loads fontspec
  \defaultfontfeatures{Scale=MatchLowercase}
  \defaultfontfeatures[\rmfamily]{Ligatures=TeX,Scale=1}
\fi
\usepackage{lmodern}
\ifPDFTeX\else
  % xetex/luatex font selection
  \setmainfont[]{ETbb}
  \setsansfont[Scale=MatchUppercase]{TeX Gyre Heros}
\fi
% Use upquote if available, for straight quotes in verbatim environments
\IfFileExists{upquote.sty}{\usepackage{upquote}}{}
\IfFileExists{microtype.sty}{% use microtype if available
  \usepackage[]{microtype}
  \UseMicrotypeSet[protrusion]{basicmath} % disable protrusion for tt fonts
}{}
% Make \paragraph and \subparagraph free-standing
\makeatletter
\ifx\paragraph\undefined\else
  \let\oldparagraph\paragraph
  \renewcommand{\paragraph}{
    \@ifstar
      \xxxParagraphStar
      \xxxParagraphNoStar
  }
  \newcommand{\xxxParagraphStar}[1]{\oldparagraph*{#1}\mbox{}}
  \newcommand{\xxxParagraphNoStar}[1]{\oldparagraph{#1}\mbox{}}
\fi
\ifx\subparagraph\undefined\else
  \let\oldsubparagraph\subparagraph
  \renewcommand{\subparagraph}{
    \@ifstar
      \xxxSubParagraphStar
      \xxxSubParagraphNoStar
  }
  \newcommand{\xxxSubParagraphStar}[1]{\oldsubparagraph*{#1}\mbox{}}
  \newcommand{\xxxSubParagraphNoStar}[1]{\oldsubparagraph{#1}\mbox{}}
\fi
\makeatother


\usepackage{longtable,booktabs,array}
\usepackage{calc} % for calculating minipage widths
% Correct order of tables after \paragraph or \subparagraph
\usepackage{etoolbox}
\makeatletter
\patchcmd\longtable{\par}{\if@noskipsec\mbox{}\fi\par}{}{}
\makeatother
% Allow footnotes in longtable head/foot
\IfFileExists{footnotehyper.sty}{\usepackage{footnotehyper}}{\usepackage{footnote}}
\makesavenoteenv{longtable}
\usepackage{graphicx}
\makeatletter
\newsavebox\pandoc@box
\newcommand*\pandocbounded[1]{% scales image to fit in text height/width
  \sbox\pandoc@box{#1}%
  \Gscale@div\@tempa{\textheight}{\dimexpr\ht\pandoc@box+\dp\pandoc@box\relax}%
  \Gscale@div\@tempb{\linewidth}{\wd\pandoc@box}%
  \ifdim\@tempb\p@<\@tempa\p@\let\@tempa\@tempb\fi% select the smaller of both
  \ifdim\@tempa\p@<\p@\scalebox{\@tempa}{\usebox\pandoc@box}%
  \else\usebox{\pandoc@box}%
  \fi%
}
% Set default figure placement to htbp
\def\fps@figure{htbp}
\makeatother





\setlength{\emergencystretch}{3em} % prevent overfull lines

\providecommand{\tightlist}{%
  \setlength{\itemsep}{0pt}\setlength{\parskip}{0pt}}



 
\usepackage[]{natbib}
\bibliographystyle{plainnat}


% FIX Marginnote duplication
\usepackage{savesym}
\savesymbol{marginfigure}
\savesymbol{marginnote}
\savesymbol{sidenote}

%%%%%%%
%  FIX Makeuppercase error
%  FIX Font clash Math error
%  See https://tex.stackexchange.com/q/202142/157312
% 

\renewcommand{\textls}[2][5]{%
  \begingroup\addfontfeatures{LetterSpace=#1}#2\endgroup
}
\renewcommand{\allcapsspacing}[1]{\textls[15]{#1}}
\renewcommand{\smallcapsspacing}[1]{\textls[10]{#1}}
\renewcommand{\allcaps}[1]{\textls[15]{\MakeTextUppercase{#1}}}
\renewcommand{\smallcaps}[1]{\smallcapsspacing{\scshape\MakeTextLowercase{#1}}}
\renewcommand{\textsc}[1]{\smallcapsspacing{\textsmallcaps{#1}}}

\PassOptionsToPackage{no-math}{fontspec}
% \usepackage[mathlf, minionint,footnotefigures, frenchmath]{MinionPro}
% \setmainfont{$$}
% \setsansfont{TeX Gyre Heros}[Scale=MatchUppercase]

\ExplSyntaxOn
\int_new:N \l_mathcode_minus_int
\int_new:N \l_mathcode_equal_int
\exp_args:Nx \AtBeginDocument {
  \exp_not:n {
    \int_set:Nn \l_mathcode_minus_int { \XeTeXmathcodenum `\- }
    \int_set:Nn \l_mathcode_equal_int { \XeTeXmathcodenum `\= }
  }
  \mathcode \int_eval:n { `\- } = \number \mathcode `\- \scan_stop:
  \mathcode \int_eval:n { `\= } = \number \mathcode `\= \scan_stop:
}
\AtBeginDocument {
  \XeTeXmathcodenum `\- = \l_mathcode_minus_int
  \XeTeXmathcodenum `\= = \l_mathcode_equal_int
}
\ExplSyntaxOff

\usepackage[italic]{mathastext}
% \setromanfont{TeX Gyre Termes}


%%%%%%%


\usepackage{pdfpages}  % for cover page
\graphicspath{{Images/}} % Make Images/ default figure path


\setlength{\parindent}{0pt}%
\setlength{\RaggedRightParindent}{0pt}
\setlength{\JustifyingParindent}{0pt}%
\setlength{\parskip}{\baselineskip}

%%
% Produces a full title page

\renewcommand{\maketitlepage}[0]{%
  \cleardoublepage%
  {%
  \sffamily%
  \begin{fullwidth}%
  \fontsize{12}{14}\selectfont\par\noindent\textcolor{darkgray}{\allcaps{\thanklessauthor}}%
  \vspace{12.5pc}%
  \fontsize{20}{28}\selectfont\par\noindent\textcolor{darkgray}{\allcaps{\thanklesstitle}}%
  \vfill%
  \fontsize{10}{12}\selectfont\par\noindent\allcaps{\thanklesspublisher}%
  \end{fullwidth}%
  }
  \thispagestyle{empty}%
  \clearpage%
}

% DEFINITIONS


% The fancyvrb package lets us customize the formatting of verbatim
% environments.  We use a slightly smaller font.
\usepackage{fancyvrb}
\fvset{fontsize=\normalsize}

%%
% Prints argument within hanging parentheses (i.e., parentheses that take
% up no horizontal space).  Useful in tabular environments.
\newcommand{\hangp}[1]{\makebox[0pt][r]{(}#1\makebox[0pt][l]{)}}

%%
% Prints an asterisk that takes up no horizontal space.
% Useful in tabular environments.
\newcommand{\hangstar}{\makebox[0pt][l]{*}}

%%
% Prints a trailing space in a smart way.
\usepackage{xspace}


% Prints the month name (e.g., January) and the year (e.g., 2008)
\newcommand{\monthyear}{%
  \ifcase\month\or January\or February\or March\or April\or May\or June\or
  July\or August\or September\or October\or November\or
  December\fi\space\number\year
}


% Prints an epigraph and speaker in sans serif, all-caps type.
\newcommand{\epigraph}[2]{%
  \begin{fullwidth}
  \begin{flushright}
  \sffamily\fontsize{8}{10}\selectfont
  \sffamily\footnotesize
  \begin{doublespace}
  \vspace{-8cm}\noindent\allcaps{#1}\\% epigraph
  \noindent\allcaps{#2}\\% author
  \end{doublespace}
  \vspace{5.1cm}
  \end{flushright}
  \end{fullwidth}
  \normalfont
}


\newcommand{\blankpage}{\newpage\hbox{}\thispagestyle{empty}\newpage}


% insert 4cm before quote
\renewenvironment{quote}{
  \list{}{\leftmargin=3.5cm\topsep=0pt}
  \item\relax\small\itshape
}
{\endlist}


%  change chapter formatting
\titlespacing*{\chapter}{0pt}{5cm}{1cm}
% \titlespacing*{\section}{0pt}{.6em}{.3em}
% \titlespacing*{\subsection}{0pt}{.4em}{.2em}

\titlespacing*{\section}{0pt}{0pt}{0pt}
\titlespacing*{\subsection}{0pt}{0pt}{0pt}


%  Change Figure Caption in the Margin size
% \renewenvironment{@tufte@margin@float}[2][-1.2ex]%
%   {\FloatBarrier% process all floats before this point so the figure/table numbers stay in order.
%   \begin{lrbox}{\@tufte@margin@floatbox}%
%   \begin{minipage}{\marginparwidth}%
%     \@tufte@caption@font\footnotesize% <-- Add fontnotesize
%     \def\@captype{#2}%
%     \hbox{}\vspace*{#1}%
%     \@tufte@caption@justification%
%     \@tufte@margin@par%
%     \noindent\normalsize%<-- restored size
%   }
%   {\end{minipage}%
%   \end{lrbox}%
%   \marginpar{\usebox{\@tufte@margin@floatbox}}%
  % }


% \renewcommand\footnotesize{%
%    \@setfontsize\footnotesize\@viiipt{9}%
%    \abovedisplayskip 5\p@ \@plus2\p@ \@minus4\p@
%    \abovedisplayshortskip \z@ \@plus\p@
%    \belowdisplayshortskip 2.8\p@ \@plus\p@ \@minus2\p@
%    \def\@listi{\leftmargin\leftmargini
%                \topsep 2.5\p@ \@plus\p@ \@minus\p@
%                \parsep 2\p@ \@plus\p@ \@minus\p@
%                \itemsep \parsep}%
%    \belowdisplayskip \abovedisplayskip
% }

% % Define Tuftian float styles (with the caption in the margin)
% \newcommand{\floatc@tufteplain}[2]{%
% \begin{lrbox}{\@tufte@caption@box}%
%   \begin{minipage}[\floatalignment]{\marginparwidth}\hbox{}%
%     \footnotesize\@tufte@caption@font{\@fs@cfont #1:} #2\par\normalsize%
%   \end{minipage}%
% \end{lrbox}%
% \smash{\hspace{\@tufte@caption@fill}\usebox{\@tufte@caption@box}}%
% }
\makeatletter
\@ifpackageloaded{bookmark}{}{\usepackage{bookmark}}
\makeatother
\makeatletter
\@ifpackageloaded{caption}{}{\usepackage{caption}}
\AtBeginDocument{%
\ifdefined\contentsname
  \renewcommand*\contentsname{Table of contents}
\else
  \newcommand\contentsname{Table of contents}
\fi
\ifdefined\listfigurename
  \renewcommand*\listfigurename{List of Figures}
\else
  \newcommand\listfigurename{List of Figures}
\fi
\ifdefined\listtablename
  \renewcommand*\listtablename{List of Tables}
\else
  \newcommand\listtablename{List of Tables}
\fi
\ifdefined\figurename
  \renewcommand*\figurename{Figure}
\else
  \newcommand\figurename{Figure}
\fi
\ifdefined\tablename
  \renewcommand*\tablename{Table}
\else
  \newcommand\tablename{Table}
\fi
}
\@ifpackageloaded{float}{}{\usepackage{float}}
\floatstyle{ruled}
\@ifundefined{c@chapter}{\newfloat{codelisting}{h}{lop}}{\newfloat{codelisting}{h}{lop}[chapter]}
\floatname{codelisting}{Listing}
\newcommand*\listoflistings{\listof{codelisting}{List of Listings}}
\makeatother
\makeatletter
\makeatother
\makeatletter
\@ifpackageloaded{caption}{}{\usepackage{caption}}
\@ifpackageloaded{subcaption}{}{\usepackage{subcaption}}
\makeatother
\makeatletter
\@ifpackageloaded{sidenotes}{}{\usepackage{sidenotes}}
\@ifpackageloaded{marginnote}{}{\usepackage{marginnote}}
\makeatother
\makeatletter
\@ifpackageloaded{bibentry}{}{\usepackage{bibentry}}
\@ifpackageloaded{marginfix}{}{\usepackage{marginfix}}
\makeatother
\makeatletter
\nobibliography*
\makeatother
\usepackage{bookmark}
\IfFileExists{xurl.sty}{\usepackage{xurl}}{} % add URL line breaks if available
\urlstyle{same}
\hypersetup{
  pdftitle={Course: CSC:XXX Course Title Documentation},
  pdfauthor={Georgia Doing},
  colorlinks=true,
  linkcolor={blue},
  filecolor={Maroon},
  citecolor={Blue},
  urlcolor={Blue},
  pdfcreator={LaTeX via pandoc}}



%  TITLE PAGE

\title[CSCXXX]{Course: CSC:XXX Course Title\\
Documentation} 
\author{Georgia Doing} 
\publisher{self}





\begin{document}
\IfFileExists{Images/bookcover.pdf}{\includepdf{Images/bookcover.pdf}\clearpage}{}

\frontmatter
\maketitle
\pagenumbering{gobble} 
\IfFileExists{01-Front/copyright.tex}{\input{01-Front/copyright.tex}\clearpage}{}
\IfFileExists{01-Front/dedication.tex}{\input{01-Front/dedication.tex}\clearpage}{}
\IfFileExists{01-Front/epigraph.tex}{\input{01-Front/epigraph.tex}\clearpage}{}
\pagenumbering{arabic}
\renewcommand*\contentsname{Contents}
{
\hypersetup{linkcolor=}
\setcounter{tocdepth}{1}
\tableofcontents
}

\mainmatter
\bookmarksetup{startatroot}

\chapter{Home}\label{home}

\noindent\includegraphics[width=\marginparwidth]{Images/et_midjourney_transparent.pdf}

The course website for \hyperref[c-desc]{CSCXXX}, part of the Union
College
\href{https://www.union.edu/computer-science/courses-requirements}{CS
curriculum}. Here you can find our wekkly schedule, assignmnets and
resources.

\section{Announcements}\label{announcements}

\subsection{Upcoming due dates}\label{upcoming-due-dates}

Next assignments is due \ldots{}

\subsection{Office Hours}\label{office-hours}

\phantomsection\label{c-desc}
\section{Course Description}\label{course-description}

\bookmarksetup{startatroot}

\chapter{Schedule}\label{schedule}

Weekly Schedule

\section{Schedule}\label{schedule-1}

\begin{longtable}[]{@{}lll@{}}
\toprule\noalign{}
Week & Topic & Assignment \\
\midrule\noalign{}
\endhead
\bottomrule\noalign{}
\endlastfoot
1 & topic\ldots{} & assignment\ldots{} \\
2 & topic\ldots{} & assignment\ldots{} \\
3 & topic\ldots{} & assignment\ldots{} \\
4 & topic\ldots{} & assignment\ldots{} \\
5 & topic\ldots{} & assignment\ldots{} \\
6 & topic\ldots{} & assignment\ldots{} \\
7 & topic\ldots{} & assignment\ldots{} \\
8 & topic\ldots{} & assignment\ldots{} \\
9 & topic\ldots{} & assignment\ldots{} \\
10 & topic\ldots{} & assignment\ldots{} \\
\end{longtable}

\bookmarksetup{startatroot}

\chapter{Resources}\label{resources}

\bookmarksetup{startatroot}

\chapter{Syllabus}\label{syllabus}

\subsection{Syllabus}\label{syllabus-1}

\includepdf[pages=-]{../Materials/CSC106_Syllabus_F2025.pdf}

\bookmarksetup{startatroot}

\chapter{Grading}\label{grading}

\begin{longtable}[]{@{}lll@{}}
\caption{Grading Scheme}\tabularnewline
\toprule\noalign{}
Course Element & Grade Point Contribution & Notes \\
\midrule\noalign{}
\endfirsthead
\toprule\noalign{}
Course Element & Grade Point Contribution & Notes \\
\midrule\noalign{}
\endhead
\bottomrule\noalign{}
\endlastfoot
Final Exam & X\% & written \\
Midterm Exam & X\% & written \\
Project & X\% & written \\
Engagement & X\% & written \\
& & \\
total & 100\% & \\
\end{longtable}

\begin{longtable}[]{@{}ll@{}}
\caption{Letter Grade Scale}\tabularnewline
\toprule\noalign{}
Letter Grade & Percent range \\
\midrule\noalign{}
\endfirsthead
\toprule\noalign{}
Letter Grade & Percent range \\
\midrule\noalign{}
\endhead
\bottomrule\noalign{}
\endlastfoot
A & 93 - 100 \\
A- & 90 - 92 \\
B+ & 87 - 89 \\
B & 83 - 86 \\
B- & 80 - 82 \\
C+ & 77 - 79 \\
C & 73 - 76 \\
C- & 70 - 72 \\
D & 60 - 69 \\
F & 0 - 59 \\
\end{longtable}

\part{Assignments}

\chapter{Project}\label{project}

\part{Weekly Notes}

\chapter{Week 1 Test}\label{week-1-test}

\chapter{Week 2}\label{week-2}

\chapter{Week 3}\label{week-3}

\chapter{Week 4}\label{week-4}

\chapter{Week 5}\label{week-5}

\chapter{Week 6}\label{week-6}

\chapter{Week 7}\label{week-7}

\chapter{Week 8}\label{week-8}

\chapter{Week 9}\label{week-9}

\chapter{Week 10}\label{week-10}

\part{about.qmd}

\chapter*{References}\label{references}

\markboth{References}{References}

\renewcommand{\bibsection}{}
\bibliography{references.bib}

\chapter*{Acknowledgements}\label{acknowledgements}

\markboth{Acknowledgements}{Acknowledgements}

Many thanks to the Quarto team for creating this wonderful tool. I am
also grateful for the support I got at
\href{https://github.com/quarto-dev/quarto-cli/discussions?discussions_q=author\%3Afredguth}{Quarto's
Discussion Board}, specialy:

\begin{itemize}
\tightlist
\item
  Mickaël Canouil (\texttt{@mcanouil});
\item
  Charles Teague (\texttt{@dragonstyle});
\item
  Raniere Silva (\texttt{@rgaiacs}); and
\item
  Christophe Dervieux (\texttt{@cderv})
\end{itemize}

\chapter*{Colophon}\label{colophon}

\markboth{Colophon}{Colophon}

Composed in ET Book, a free and open-sopurce typeface designed by Dmitry
Krasny, Bonnie Scranton, and Edward Tufte; and \TeX  Gyre Heros, a
fontbased on the URW Nimbus Sans L kindly released by URW++ Design and
Development Inc.~under GFL. Based on
\href{https://tufte-latex.github.io/tufte-latex/}{Tufte-Latex} and
\href{https://quarto.org}{Quarto}.

{\marginnote{\begin{footnotesize}\pandocbounded{\includegraphics[keepaspectratio]{Images/gutemberg_press.pdf}}\end{footnotesize}}}


\backmatter





\end{document}
